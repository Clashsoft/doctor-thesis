% \chapter*{Exposé}\label{ch:expose}

\section{Motivation}\label{sec:motivation}

Die Entwicklung von modernen Webanwendung erfordert die Kenntnis einer Vielzahl von Konzepten, Techniken, Werkzeugen und Frameworks.

\section{Zielsetzung}\label{sec:zielsetzung}

Die Arbeit soll verschiedene Aspekte der Webentwicklung beleuchten und Best Practices, Werkzeuge und Bibliotheken bereitstellen, um diese Aspekte handhabbar und einsteigerfreundlich zu machen.
Zunächst soll geklärt werden, welche existierenden und bewährten Frameworks eingesetzt werden können, um eine Grundlage zu bieten.
Überlegungen zu klassisch getrennten Komponenten wie Frontend, Backend und Datenbank sollen im Kontext moderner Full-Stack-Entwicklung stets im Verbund geschehen.
Dadurch wird die Kopplung erhöht, aber es kann doppelte Entwicklungsarbeit (sog. Boilerplate-Code) und Absprachen vermieden werden.
Ziel ist es, Webanwendungen für Anfänger sowie kleine und mittelständische Unternehmen (KMU) handhabbar zu machen und eine klare Abgrenzung zu großen Anwendungen und Microservices zu schaffen.
Als beispielhafter Technologie-Stack wird zunächst der MEAN-Stack (MongoDB, Express, Angular, Node.js) betrachtet.
Dabei soll auf folgende Aspekte in den jeweiligen Komponenten eingangen werden:

\begin{itemize}
    \item Projekteinrichtung (Nx)
    \item Projektstruktur (Nx)
    \item Ressourcenstruktur (REST)
    \item Frontend (Angular)
    \begin{itemize}
        \item REST und GraphQL
        \item Aktualisierung (SSE, WebSocket)
        \item Formulare (insb.\ Validierung)
        \item Listenansichten (insb.\ Filterung, Sortierung, Paginierung, Aktualisierung, Lazy Loading)
        \item Detailansichten (insb.\ Aktualisierung, Kollaboration)
        \item Authentifizierung (JWT)
        \item Autorisierung (insb.\ Rollen, Rechte)
        \item SEO (Angular Universal, insb.\ Prerendering)
        \item PWA und Offline (Service Worker)
        \item Monitoring (Sentry)
        % \item Internationalisierung
        % \item Logging
        % \item Performance (Lighthouse)
        % \item Accessibility (Lighthouse)
        % \item Testing (insb. Unit, E2E)
        % \item Deployment (insb. Docker, Kubernetes)
        % \item Dokumentation (insb. Compodoc)
        % \item CI/CD (insb. GitHub Actions)
        % \item Security (z.B.\ CORS, CSP, CSRF, XSS, XSRF)
        % \item Micro Frontends
        % \item State Management (insb.\ NgRx)
    \end{itemize}
    \item Backend (NestJS)
    \begin{itemize}
        \item Authentifizierung (JWT)
        \item Autorisierung (insb.\ Rollen, Rechte)
        \item Datenbankanbindung (Mongoose)
        \item Caching (Redis)
        \item Messaging und Microservices (NATS)
        \item Monitoring (Sentry)
        \item Skalierung (insb.\ Load Balancing)
        % \item Logging
        % \item Security (z.B.\ CORS, CSP, CSRF, XSS, XSRF)
        % \item Testing (insb.\ Unit, E2E)
        % \item Deployment (insb.\ Docker, Kubernetes)
        % \item Dokumentation (insb.\ Compodoc)
        % \item CI/CD (insb.\ GitHub Actions)
    \end{itemize}
    \item Datenbank (MongoDB)
\end{itemize}

\section{Zeitplan}\label{sec:zeitplan}

\autoref{tbl:timetable} zeigt den geplanten Zeitplan für die Arbeit.
Im ersten Jahr sollen die Grundlagen recherchiert und mit der Konzeption begonnen werden.
Als Grundlage dienen Expertengespräche, Befragungen und Lehrveranstaltungen.
Im zweiten Jahr soll die Konzeption abgeschlossen und die Evaluation durchgeführt werden.
Darunter fällt auch die Implementierung von Protypen, Bibliotheken und Werkzeugen.
Im dritten Jahr soll die Arbeit verschriftlich und abgeschlossen werden.
Der Zeitplan ist vorläufig und kann sich im Laufe der Arbeit ändern.

\begin{landscape}
    \begin{table}
        \centering
        \caption{Zeitplan}
        \begin{tabular}{|l|l|l|l|l|l|l|l|l|l|l|l|l|}
        \hline
            & \multicolumn{4}{c|}{Jahr 1}  & \multicolumn{4}{c|}{Jahr 2} & \multicolumn{4}{c|}{Jahr 3} \\ \hline
            Monat  & 1 - 3 & 4 - 6 & 7 - 9 & 10 - 12 & 1 - 3 & 4 - 6 & 7 - 9 & 10 - 12 & 1 - 3 & 4 - 6 & 7 - 9 & 10 - 12 \\ \hline
            \cellcolor{yellow!75}Orientierungsphase & \cellcolor{yellow!75} & \cellcolor{yellow!75} &  &  &  &  &  &  &  &  &  &  \\ \hline
            ~ Grobe Literatursichtung  & \cellcolor{yellow!35} &  &  &  &  &  &  &  &  &  &  &  \\ \hline
            ~ Vorläufige Gliederung aufstellen  & \cellcolor{yellow!35} & \cellcolor{yellow!35} &  &  &  &  &  &  &  &  &  &  \\ \hline
            \cellcolor{yellow!75}Recherchephase  &  & \cellcolor{yellow!75} & \cellcolor{yellow!75} & \cellcolor{yellow!75} &  &  &  &  &  &  &  &  \\ \hline
            ~ Intensive Literaturrecherche  &  & \cellcolor{yellow!35} & \cellcolor{yellow!35} &  &  &  &  &  &  &  &  &  \\ \hline
            ~ Material sammeln  &  &  & \cellcolor{yellow!35} & \cellcolor{yellow!35}  &  &  &  &  &  &  &  &  \\ \hline
            \cellcolor{yellow!75}Konzeptionsphase  &  &  &  & \cellcolor{yellow!75} & \cellcolor{yellow!75} &  \cellcolor{yellow!75} &  &  &  &  &  &  \\ \hline
            ~ Konzept entwickeln  &  &  &  & \cellcolor{yellow!35} & \cellcolor{yellow!35} &  &  &  &  &  &  &  \\ \hline
            ~ Expertengespräche / Befragungen  &  &  &  &  & \cellcolor{yellow!35} & \cellcolor{yellow!35} &  &  &  &  &  &  \\ \hline
            ~ Konzept verfeinern  &  &  &  &  &  &\cellcolor{yellow!35}  &  &  &  &  &  &  \\ \hline
            \cellcolor{yellow!75}Evaluationsphase  &  &  &  &  &  &  &  \cellcolor{yellow!75} &  \cellcolor{yellow!75} & \cellcolor{yellow!75} &  &  &  \\ \hline
            ~ Nutzertests  &  &  &  &  &  &  & \cellcolor{yellow!35} & \cellcolor{yellow!35} &  &  &  &  \\ \hline
            ~ Auswertung  &  &  &  &  &  &  &  & \cellcolor{yellow!35} & \cellcolor{yellow!35} &  &  &  \\ \hline
            \cellcolor{yellow!75}Schreibphase  &  &  &  &  &  &  &  &  & \cellcolor{yellow!75} & \cellcolor{yellow!75} & \cellcolor{yellow!75} &  \\ \hline
            ~ Grundlagen beschreiben  &  &  &  &  &  &  &  &  & \cellcolor{yellow!35} &  &  &  \\ \hline
            ~ Konzept beschreiben  &  &  &  &  &  &  &  &  & \cellcolor{yellow!35} & \cellcolor{yellow!35} &  &  \\ \hline
            ~ Evaluation beschreiben  &  &  &  &  &  &  &  &  &  & \cellcolor{yellow!35} & \cellcolor{yellow!35} &  \\ \hline
            ~ Fazit / Ausblick verfassen  &  &  &  &  &  &  &  &  &  &  & \cellcolor{yellow!35} &  \\ \hline
            \cellcolor{yellow!75}Abschlussphase  &  &  &  &  &  &  &  &  &  &  & \cellcolor{yellow!75} & \cellcolor{yellow!75} \\ \hline
            ~ Korrekturlesen  &  &  &  &  &  &  &  &  &  &  & \cellcolor{yellow!35} &  \\ \hline
            ~ Fehler beheben  &  &  &  &  &  &  &  &  &  &  & \cellcolor{yellow!35} & \cellcolor{yellow!35} \\ \hline
            ~ Veröffentlichen  &  &  &  &  &  &  &  &  &  &  &  & \cellcolor{yellow!35} \\ \hline
            Monat  & 1 - 3 & 4 - 6 & 7 - 9 & 10 - 12 & 1 - 3 & 4 - 6 & 7 - 9 & 10 - 12 & 1 - 3 & 4 - 6 & 7 - 9 & 10 - 12 \\ \hline
            & \multicolumn{4}{c|}{Jahr 1}  & \multicolumn{4}{c|}{Jahr 2} & \multicolumn{4}{c|}{Jahr 3} \\ \hline

        \end{tabular}
        \label{tbl:timetable}
    \end{table}
\end{landscape}

% \section{Literatur}\label{sec:literatur}

\nocite{*}
